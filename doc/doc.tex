\documentclass[12pt]{article}

\usepackage{sbc-template}

\usepackage{graphicx,url}

\usepackage[brazil]{babel}   
%\usepackage[latin1]{inputenc}  
\usepackage[utf8]{inputenc}  
% UTF-8 encoding is recommended by ShareLaTex

\sloppy

\title{O SUPER JOGO!!!}

\author{Daniel A. Amarante, Diego P. da Jornada}


\address{Faculdade de Informática -- Pontifícia Universidade Católica do Rio
    Grande do Sul\\  (PUCRS)\\
  \email{daniel.amarante, diego.jornada}@acad.purs.br
}

\begin{document} 

\maketitle

\begin{abstract}
  This paper was written as the documentation of the first project of the Computer Graphics I class of PUCRS.
  It describes a shooter type game about a lonely Lollipop ship flying through space while fighting the evil
  cookies and nachos. The game was written in C++ and uses OpenGL for graphics.
\end{abstract}
     
\begin{resumo} 
  Este artigo foi escrito como a documentação do primeiro projeto da disciplina de Computação Gráfica I da PUCRS.
  Ele descreve um jogo de estilo Shooter sobre um pirulito solitário voando através do espaço enquanto luta com os
  biscoitos e nachos malignos. O jogo foi escrito em C++ e utiliza do OpenGL para os gráficos.
\end{resumo}

\section{Informações gerais sobre o jogo}

O jogo consiste de um \emph{Shooter} 2d situado no espaço sideral cujo objetivo é sobreviver à onda de inimigos que 
se encontrarão no caminho. O jogador dispõe da possibilidade de se mover para os lados, sendo limitado pelo fim da tela,
para desviar dos inimigos e da possibilidade de atirar contra eles. Um tiro, ao entrar em contato com um inimigo,
o destrói, destruindo a si mesmo como consequência. Para vencer o jogo, o jogador deve chegar até o final do mapa
sem ser pego pelos inimigos. Para ser derrotado basta um inimigo colidir com o jogador.

\section{Projeto de Classes}

Todos os objetos do jogo herdam de uma classe chamada GameObject, com funções que serão herdadas por todos, como as
funções de movimentação e as funções de desenhar, a movimentação é igual para todos os objetos que se movimentam, mas
a forma com que são desenhados vai ter que ser sobrescrita.

\section{Formas}

Os objetos do jogo assumem diversas formas, o jogador consiste de um circulo e um retângulo, para se assemelhar a um
pirulito, os inimigos são formados por circulos e triângulos, representando biscoitos e nachos respectivamente. E o
mapa ao fundo consiste de diversas estrelas desenhadas em posições aleatorias.

\section{Movimentação}

O único objeto que avança no jogo é o jogador, trazendo consigo a tela. Todos os inimigos somente se encontram no
caminho, movendo-se para os lados. Existe a impressão de que os inimigos é que vem em direção ao jogador, e o jogador
está parado, mas é apenas uma ilusão causada pelo movimento da tela junto com o movimento do player.

\section{Colisões}

As colisões do jogo foram implementadas da seguinte forma. Desenha-se uma caixa em volta dos objetos, utilizando
seus pontos mais extremos, que será utilizada para detectar a colisão. Desta caixa tiramos seus vértices. É possível
saber se um ponto se encontra dentro de uma caixa, comparando-o com seus vértices. Portanto, utilizamos desta comparação
nos quatro vértices do objeto 1, para saber se se encontram dentro do retângulo do objeto 2, caso a resposta seja
verdadeira temos uma colisão.

\bibliographystyle{sbc}
\bibliography{doc}

\end{document}
